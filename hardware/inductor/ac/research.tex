\documentclass{article}

\usepackage{amsmath}
\numberwithin{equation}{subsection}

\title{Analyzing induction from an insulated wire carrying AC current}
\author{Carson Derr <cakub6@gmx.com>}
\date{August 2022}
\begin{document}
\maketitle

\section{Introduction}
The goal here is to experiment with various designs for maximizing the inductance
from an insulated wire carrying a typical residental AC voltage and current.

\section{Models}
Unless otherwise noted, all interal wire resistance will be ignored in any of the following models.

\subsection{Current Carrying Wire}

The magnetic field around a wire carrying an alternating current can be written as:

\begin{equation}
\vec{B}(r, t) = \frac{\mu_0 i(t) \cos(\omega t + \varphi)}{2 \pi r}\vec{\phi}
\end{equation}

\subsection{Single Loop Inductor}

\subsection{Toroidal Inductor}

A toroidal inductor can be constructed by tightly wrapping wire around a torus (aka donut). A current
carrying wire can then be placed azimuthal in the center of the torus causing a magnetic field to "flow"
concentrically through the inductor.

The magnetic flux within the inductor can then be computed using Equation 2.1.1 and stating
that the loops of the toroid can be approximated as the sum of \(N\) number of loops each with an area
given by the cross-sectional area of the torus.

\subsubsection{Rectangular Toroid}

Rotating a rectangle a distance, \(R_I\), around an axis parallel to one of its edges produces a rectangular toriod. If
wire is tighly wrapped around this toroid so it can be thought of as \(N\) number of rectangular slices,
then the magnetic flux from a current carrying wire placed along its axis of rotation is the following: 

\begin{equation}
\begin{split}
\Phi_B(t) & = \int \vec{B}(r, t) \cdot \, \vec{dA} \\
  & = \int B(r, t) \, dA \\
  & = \frac{\mu_0 N b}{2 \pi} i(t) \cos(\omega t + \varphi) \int_{R_I}^{R_I + a} \frac{1}{r} \, dr \\
  & = \frac{\mu_0 N b}{2 \pi} \ln(\frac{R_I + a}{R_I}) i(t) \cos(\omega t + \varphi) \\
  & = \frac{\mu_0 N b}{2 \pi} \ln(1 + \frac{a}{R_I}) i(t) \cos(\omega t + \varphi)
\end{split}
\end{equation}

Now, having computed the flux through through the toriod the induced emf can be computed.

\begin{equation}
\begin{split}
E(t) = \frac{\mu_0 N b}{2 \pi} \ln(1 + \frac{a}{R_I}) [ \omega i(t) \sin(\omega t + \varphi) - \frac{di}{dt} cos(\omega t + \varphi)]
\end{split}
\end{equation}

Assuming the current amplitude is constant, the induced emf can be simplified to:

\begin{equation}
\begin{split}
E(t) = \frac{\mu_0 \omega N I b}{2 \pi} \ln(1 + \frac{a}{R_I}) \sin(\omega t + \varphi)
\end{split}
\end{equation}


\subsection{Concentric Toroidal Inductor}

\section{Experiments}

\subsection{Single Loop Inductor}

\subsection{Toroidal Inductor}

\subsection{Concentric Toroidal Inductor}

\end{document}